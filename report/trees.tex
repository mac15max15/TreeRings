

\documentclass[]{article}

%opening
\title{trees wip}
\author{Max Caragozian}

\begin{document}

\maketitle

\section{Preface}
The  American West has the largest, tallest, and oldest trees in the world. Those superlative trees, however, are often found in environments that are harsh either due to their aridity or their bitter winters. While the high mountains of the Sierra Nevada and the Transverse Ranges have consistent yearly precipitation in the form of winter snow, plants lower down depend on fickle rains that sometimes fail for years on end.

In the home of the world's oldest trees, tree rings encode that climate in a history stretching back almost two thousand years. Interpreting that history and using it to make predictions about the future is of huge practical value, as California has the largest agricultural output of any state. Sierra Nevada snowmelt trapped in resevoirs and Colorado River water pumped in by aqueduct irragate most of those farms. Tree rings give climate scientists a window into how the snowpack and rivers looked hundreds of years in the past, which is only becoming more important in the era of climate change.

I grew up in Los Angeles and spent weekends and summers tromping around the natural wonderland that California has to offer. I gained an appreciation for the sheer variety of climates and ecosystems the West contains. Chaparall with regal old oaks. Subalpine forests with fir and pine. Alpine tundra where krumholzed lodgepole pines cling to the edge of the tree line. All can be found within thirty miles of Los Angeles, and each biome brings back warm memories for me. I learned very quickly that in Southern California, a good year for plants is synonymous with a wet year. I aim to use tree ring and historical data to quantify that relationshoip and investigate how climate change might affect it. 

\section{Sources and Locations}



\end{document}
